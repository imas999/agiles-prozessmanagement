\section{Generationsüberblick}

\begin{table}[h!]
    \centering
    \begin{tabular}{c|c|c|c|c}
        \textbf{Generation} & \textbf{Jahr} & \textbf{Datenrate} & \textbf{Modulation} & \textbf{Frequenzband} \\
        \hline
        1G & ab 1978 & --- & --- & kein Spezielles \\
        \hline
        2G (GSM) & ab 1990 & max. 9,6 kbit/s & GMSK & 900 MHz, 1800 MHz \\
        \hline
        3G (UMTS) & ab 2003 & 384 kbit/s & QPSK & 1920--2170 MHz \\
        \hline
        4G (LTE) & ab 2012 & 300 MBit/s & 64-QAM & 700--2700MHz \\
        \hline 
        5G & ab 2019 & 10 GBit/s & 256-QAM &  700MHz, 3.5GHz, 24GHz \\
    \end{tabular}
    \caption{Übersicht Technische Daten der Mobilfunkgenerationen \cite{uebersicht}}
    \label{tab:generationen}
\end{table}

\noindent Um eine schnelle Übersicht über die verschiedenen Mobilfunkgenerationen zu erhalten, ist in Tabelle \ref{tab:generationen} zunächst eine kurze Zusammenfassung der wichtigsten technischen Kenngrößen dargestellt.
Ergänzend zu Tabelle \ref{tab:generationen} ist zu erwähnen, dass die verschiedenen Mobilfunkgenerationen unterschiedliche Aufgabenbereiche abdecken. Die erste Generation ist hierbei die ersten Anläufe der Telefonie über analgoe Kabelnetze.
Darauf aufbauen mit dem Global System for Mobile Communications (GSM) Standard wurde die Übertragung von SMS und einfache mobile Kommunikation eröglicht. Mit dem UMTS Standard der dritten Generation wurde die mobile Datenübertragung, also das mobile Internet eingeführt.
Durch den im Jahr 2012 beschlossenen \glqq Long Term Evolution\grqq (LTE) Sandard kam nun ein verbesserte Datenrate, flächendeckenden guten Empfang mit einer deutlich geringeren Latenzzeit hinzu. Aufgrund der hohen Nachfrage und der immer digtaler 
ausgelegten Gesellschaft wurde 2016 der neue 5G Standard eingeführt. Hierbei wird vor allem die Datenrate drastisch erhöht, mit bis zu dem zehnfachen des LTE Standards. Ein interessanter Aspekt bei der Betrachtung der verschiedenen Mobilfunkgenerationen ist deren Lebensdauer.
\begin{wrapfigure}{r}{0.4\textwidth}
  \centering
  \includegraphics[width=0.38\textwidth]{images/lebenszyklen.png}
  \caption{Lebenszyklen \cite{uebersicht}}
  \label{fig:lebenszyklen}
\end{wrapfigure} 
Die verschiedenen Lebenszyklen sind in Abbildung \ref{fig:lebenszyklen} visuell dargestellt. Zu beachten ist, dass sich der GSM Standard viel länger gehalten hat, als sein eigentlicher Nachfolger UMTS. Zudem erwähnenswert ist, dass immer eine gewisse Zeitliche überlappung
der einzelnen Generationen besteht, da die Infrastruktur und die Geräte nicht sofort auf den neusten Stand gebracht werden können. \\
Eng mit den Lebenszyklen verknüpft, ist die Verfügbarkeit und die Erschwinglichkeit für die Breite Masse. Gerade zu Beginn der Telefonie war dies nur den wohlhabenderen Einwohner vorbehalten, mit Preisen bis zu 15.000 DM. Ihren Höhepunkt erreicht die analoge Telefonie mit der Einführung
des C-Netzes und einem Höchststand von 800.000 Nutzern \cite{fg_funk}. Mit der Einführung des ersten digitalen Mobilfunkstandards GSM, wurde der wohl prägenste Schritt in der Mobilkommunikation getroffen, der sich bis heute noch aktiv genutzt wird.
Mit über 76 Millionen Nutzern in Deutschland (Stand Ende 2005 \cite{fg_funk}), wurde hiermit erstmals die Breite Masse der Bevölkerung abgedeckt.
Auch wenn der neue UMTS Standard eine deutlich bessere Datenrate im vergleich zu GSM bot, konnte sich dieser nicht durchsetzen. Der UMTS Standard wurde bereits im Jahr 2021 abgestellt, währenddessen der GSM Standard weiterhin für die flächendeckende Grundversorgung in Deutschland zuständig ist. 
Gründe dafür sind zum einen der Frequenzbedarf. Die Frequenzen die UMTS nutzte werden derzeit für die Ausbau des 4G und 5G Netzes gebraucht. Zum anderen ist die die Flächendeckung durch GSM bereits gegeben und auch Systeme wie das Notrufsystem beruht noch auf GSM.
Wie auch schon in der Einführung erwähnt, ist der Ausbau des 4G und 5G Standards in Deutschland eher schleppend. Nach einem Ranking von der Strategie- und Innovationsberatung Arthur D. Litte aus dem Jahr 2019 \glqq liegt Deutschland zwischen Lettland und Norwegen auf dem 18. von 43 Plätzen\grqq \cite{falck_5g}.
Als mögliche Gründe dafür, kann mitunter zwar auch die Politik und die Netzbetreiber aufgeführt werden, allerdings ist es Gerade in Deutschland durch das gegebene Terrain und die weniger geballten bewohnten Flächen eine Herausforderung, flächendeckend und den größtmöglichen Teil der Bevölkerung abzudecken. \\
Um noch einmal auf die Überischt aus Tabelle \ref{tab:generationen} zurückzukommen, ist es offensichtlich in einem Generationenvergleich auch auf die für den Endnutzer am wichtigste Kenngröße, die Datenrate zu schauen. Hierbei ist allerdings auch den den Verwendungszweck und die Zeitepoche zu berücksichtigen.
In einer Zeit, in der Streamingdienste und digtiale Kommunikation immer relevanter sind, ist eine höhere Datenrate nötig im Vergleich zu den Beginn des GSM Standards, in welcher lediglich einfache SMS verschickt werden sollten. Dennoch sind die Sprünge, gerade der von UMTS auf LTE und der von LTE auf 5G mit bis zu der 10fachen Datenrate positiv hervorzuheben. \\
Als letzter Vergleichspunkt wird die genutzte Modulationsart im Bezug auf deren Signal-zu-Rausch-Verhältnis (SNR) für eine bestimmte Bitfehlerrate (BER) betrachtet. 
\begin{table}[h!]
  \centering 
  \begin{tabular}{c|c|c}
    \textbf{Modulation} & \textbf{SNR für BER = $10^{-6}$} & \textbf{Referenz} \\
    \hline
    GMSK & 8 - 9dB (für BER = $10^{-3}$ ) & \cite{singh_modulation} \\
    \hline
    QPSK & 9.5 - 10dB & \cite{masud_ber} \\
    \hline
    64-QAM & 18 - 22dB & \cite{singh_modulation} \\
    \hline
    256-QAM & 25 - 30dB & \cite{singh_modulation} \\ 
  \end{tabular}
  \label{tab:mod_snr_ber}
  \caption{Übersicht SNR für verschiedene Modulationsarten bei BER = $10^{-6}$}
\end{table}
\noindent Hierbei beschreibt das SNR das Verhältnis zwischen Signalleistung und Rauschleistung. Einfach ausgedrückt, je höher das SNR, desto besser ist die Signalqualität.
Die hier referenzierte Bitfehlerrate (BER) ist ein Maß, wie viele Bits falsch sind, im Verhältnis zu den gesamten gesendeten Bits. Hier gilt, je kleiner das BER, desto besser die Singalqualität. 
Tabelle \ref{tab:mod_snr_ber} zeigt deutlich, den Zusammenhang zwischen der Anzahl der verfügbaren Bits und der Notwendigkeit für robustheit gegenüber Störungen wie Rauschen. 
Daraus abzuleiten ist ebenfalls, dass die Technik hinter den Sendern und Empfänger immer besser und robuster werden muss um die immer Höheren Datenraten zu unterstützen.

\section{Forschung - Zukunftsausblick}
Natürlich wird derzeit von vielen verschiedenen Instituten an neuen 6G Standards geforscht. Spezieller Fokus liegt hierbei vor allem auf den europäischen Zielsetzugen, beschrieben im White Paper \glqq European vision for the 6G network ecosystem\grqq \cite{6GIA2024}.
Als die wichtigsten Punkte, neben dem immer existenten Ziel der Nachhaltigkeit, ist vor allem eine schnelleren Latenzzeit (von $<1ms$) und eine perfekte Flächenabdeckung, durch eine Kombination aus Basisstationen und Satelliten. Zusätzlich sollen Peak-Datenraten von bis zu 1Tbis/s, durch Nutzung von Frequenzen im Terahertzbereich, ermöglicht werden.
Gerade im Automobilbereich und der Industrie werden Themen wie \textbf{Edge-Computing} und \textbf{Zero-Trust-KI-Architekturen}, die massiv von den Vorteilen der 6G Standards profitieren, einen großen Einfluss auf die geplanten Standards haben.

