\section{Zusammenfassung}
Zusammenfassend lässst sich sagen, dass gerade in den letzten zwei Jahrzehnten große Sprünge in der Verfügbarkeit und der Effizienz der Mobilfunksysteme gemacht wurden.
Von teuren einzelnen analogen Endgreäten, hin zu flächendeckenden digitalen und schnellen Mobilfunk. Dennoch ist die Nachfrage und Forschung zu immer schnelleren 
und noch effizienteren Mobilfunksystemen so potent wie nie zuvor. Gerade in Deutschland wird der 5G Ausbau stark gefördert. Allerdings kommen hier auch Gegenstimmen, wie CDU-Ministerin Karliczek auf. Diese sagte in einem Interview mit der FAZ: \glqq 5G nicht an jeder Milchkanne notwendig\grqq \cite{Karliczek2018}. 
Gemeint ist hiermit, dass nicht jede Region in Deutschland unbedingt mit 5G versorgt werden muss, was eine weiterführende Debatte auch in Zukunft über die Verfügbarkeit von 6G eröffnet.