\section{Grundlagen}
Um die technischen Unterschiede der verschiedenen Mobilfunksysteme zu verstehen, ist es wichtig, einige grundlegende Konzepte der Nachrichtentechnik zu kennen. 
Dazu zählt das wichtigste mathematische Werkzeug zur Analyse von Signalen, die Fourier-Transformation. Hierbei geht es darum, die Frequenzanteile einzelner Signale zu Analysieren um zum Beispiel die Bandbreite zu bestimmen. 
Relevant wird dies vor allem bei der Nutzung von Funkbändern, da Unternehmen nur begrenzte Frequenzbereiche zur Verfügung haben. Auch bei der Modulation von Signalen ist die Fourier-Transformation hilfreich,
da hier Informationen in Wellenformen codiert werden.
%Speziell sind die folgenden Konzepte relevant: 
%\begin{itemize}
%    \item \textbf{Fourier-Analyse und Spektrum}: Mathematisches Werkzeug zur Zerlegung von Signalen in ihre Frequenzanteile.
%    \item \textbf{Funkbänder}: Vordefinierte Frequenzbereiche, die von Mobilfunksystemen genutzt werden.
%    \item \textbf{Modulation}: Vorgang, bei dem Informationen in eine Wellenform zur Übertragung codiert werden.
%    \item \textbf{Zellularkonzept}: Struktur des Mobilfunknetzes zur effizienten Nutzung von Frequenzen. 
%\end{itemize}

\subsection{Diskrete Fourier-Transformation und Spektrum}
Um zunächst zu verstehen, warum die Fourier-Analyse beziehugnsweise die Fourier-Transformation und das dazugehörige Spektrum wichtig für die Signalübertragung ist, muss zunächst geklärt werden, was mit einem komplexen zeitdiskreten Signal gemeint ist.
Bei einem komplexen Signal handelt es sich um ein Signal, welches sowohl einen Realteil als auch einen Imaginärteil besitzt:
\begin{equation}
    s(t) = I(t) + j \cdot Q(t)
\end{equation}
Hierbei bezeichnet der Realteil $\Re{s(t)} = I(t)$ den In-Phase-Anteil und der Imaginärteil $\Im{s(t)} = Q(t)$ den Quadratur-Anteil des Signals. 
Wenn nun dieses Signal zeitdiskret ist, bedeutet dies, dass es nur zu konkreten Zeitpunkten definiert ist, also in der Regel, ein Signal, welches durch Abtastung eines kontinuierlichen Signals entsteht.\\
Relevant für die Signalübertragung ist nun die Tatsache, dass komplexe zeitdiskrete Signale durch die Fourier-Transformation in ihre Frequenzanteile zerlegt werden können. Dadurch ist es möglich, die Modulation einfacher durchzuführen 
und die Eigenschaften wie die Bandbreite und die Frequenznutzung zu analysieren:
\begin{equation}
    S[k] = \sum_{n=0}^{N-1} s[n] \cdot e^{-j \frac{2 \pi}{N} k n} \quad \text{für } k = 0, 1, \ldots, N-1   
\end{equation}

\subsection{Beispiel: Diskrete Fourier-Transformation eines Cosinuses}
Um das abstrakte Konzept der Diskreten Fourier-Transformation besser zu verstehen, wird im folgenden ein einfaches grafisches Beispiel gezeigt.
Anhand einer Cosinusfunktion mit einer Frequenz von 10kHz ist der nutzen der DFT zur Analyse der Frequenzanteile dargestellt:
\begin{equation}
  s(t) = \cos(2 \cdot \pi \cdot 10.000 \cdot t) \\ \laplace S(f) = \frac{1}{2} \left[ \delta(f - 10.000) + \delta(f + 10.000) \right] 
\end{equation}
Grafisch dargestellt sieht dies wie folgt aus:
\begin{figure}[h!]
    \centering
    \includegraphics[width=0.5\textwidth]{images/cosinus_10khz.png}
    \caption{Diskrete Fourier-Transformation eines 10kHz Cosinussignals}
    \label{fig:dft_sinus}
\end{figure}
\\ In der Abbildung \ref{fig:dft_sinus} ist im oberen Teil das Cosinussignal im Zeitbereich dargestellt und im unteren Teil das dazugehörige Spektrum. 
Relevant ist, dass das Spektrum zwei Peaks bei $\pm 10kHz$ aufweist, genau die Frequenz des ursprünglichen Cosinussignals.

%\subsection{Funkbänder}
%\begin{wrapfigure}{h!}{0.5\textwidth}
%  \begin{center}
%    \includegraphics[width=0.48\textwidth]{images/frequenzpolitikgrafik.jpg}
%  \end{center}
%  \caption{Frequenzplan Deutschland Stand 2022 \cite{frequenznutzung}}
%  \label{fig:frequenzplan}
%\end{wrapfigure}
%Der Mobilfunk in Deutschland nutzt verschiedene Frequenzbereiche, die sogenannten Funkbänder.

\subsection{Modulation}
Wenn Informationen in Form von Bits übertragen werden sollen, müssen diese zunächst zu Symbolen zusammengefasst werden. 
Bei diesen Symbolen handelt es sich um diskrete Einheiten, die schlussendlich in Wellenformen codiert werden müssen. 
Wenn mann von einer m-wertigen Modulation spricht, bedeutet dies, dass es insgesamt $m = 2^k$ Symbole mit k Bits pro Symbol gibt. \\
Nun ist es so, dass bei den verschiedenen Mobilfunkgenerationen unterschiedliche Anforderungen an die Effizienz, die Fehleranfälligkeit und die Komplexität gestellt werden. 
Deshalb ist es wichtig, die grundlegenen Modulationsverfahren zu kennen, die in der verschiedenen Mobilfunkgenerationen verwendet werden. 

\subsubsection{Frequenzmodulation (FM)}
Bei der Frequenzmodulation handelt es sich um ein analoges Modulationsverfahren. Hierbei wird die Frequenz der Trägerwelle im Verhältnis zur Amplitude $A_c$ des Informationssignals variiert.
Mathematisch Beschrieben wird dies folgednermaßen:
\begin{equation}
  s(t) = A_c \cdot \cos(\omega_c \cdot t + \beta \cdot \sin(\omega_m \cdot t))
\end{equation}
Bei $\beta$ handelt es sich um den Modulationsindex, welcher die Abweichung der Frequenz vom Mittelwert beschreibt. 
Besonders durch die Einfachheit und der im Vergleich zu anderen analogen Modulationsverfahren hohe Störfestigkeit ist die Frequenzmodulation für die Audioübertragung geeignet.\cite{frequenzmodulation}

\subsubsection{Gaussian Minimum Shift Keying (GMSK) \cite{dig_nue}}
Das Gaussian Minimum Shift Keying (kurz GMSK) ist ein Sonderfall des Frequency-Shift Keying (kurz FSK), genauer gesagt eine 2-FSK. Die Symbole werden hier durch zwei verschiedene Vektoren im Konstellationsdiagramm dargestellt.
Zu beachten ist, dass die beiden Vektoren orthogonal, als in einem rechten Winkel zueinander stehen müssen:
\begin{enumerate}
  \item Symbol 0: $s_1 = A \cdot \cos(2 \pi \cdot f_1 \cdot t)$
  \item Symbol 1: $s_2 = A \cdot \cos(2 \pi \cdot f_2 \cdot t)$
\end{enumerate}
Der Linienabstand $\Delta f$ zwischen den beiden Frequenzen $f_1$ und $f_2$ ist so gewählt, dass der kleinste Modulationsabstand $\eta$ für die orthogonalität erreicht wird:
\begin{equation}
  \eta = \frac{\Delta f}{f_s}
\end{equation}
Für dem Fall der GMSK-Modulation in der Modulationsabstand $\eta = \frac{1}{2}$. Verwendung findet GSMK hauptsächlich bei robusten Verfahren mit einer kleinen Symbolrate $f_s$ ($< 10kBaud$).

\subsubsection{Phase Shift Keying (PSK)}
\begin{wrapfigure}{h!}{0.4\textwidth}
  \begin{center}
    \includegraphics[width=0.48\textwidth]{images/Konstellationsdiagramm_4psk.png}
  \end{center}
  \caption{Beispiel einer QPSK-Konstellation}
  \label{fig:psk_konstellation}
\end{wrapfigure}
Um das Prinzip der Phasenmodulation zu erklären, wind im Folgenden das Beispiel einer QPSK-Modulation betrachtet. Bei dieser Moudlation werden 4 Symbole mit jeweils 2 Bits pro Symbol codiert.
Allgemein gilt, dass bei einer M-PSK $M = 2^k$ Symbole mit jeweils k Bits pro Symbol codiert werden, mit jeweils $2^k$ unterschiedlichen Wellenformen:
\begin{equation}
  s_m(t) = A \cdot \cos(2 \cdot \pi \cdot f_0 \cdot t + \frac{2 \cdot \pi}{M} \cdot m)
\end{equation} \\
Der Vorteil an einer komplexwertigen Amplitudenmodulation wie der PSK ist, dass die inforamtion in den Koeffizienten der Linearkombination der Basisfunktionen codiert wird. Durch eine zusätzliche Zuordnung der Bits im Gray-Code 
(nur ein Bit ändert sich pro Symbolwechsel) kann die Bitfehlerwahrscheinlichkeit weiter reduziert werden, wobei die Symbolfehlerwahrscheinlichkeit konstant bleibt. 

\subsubsection{Quadrature Amplitude Modulation (QAM)}
\begin{wrapfigure}{r}{0.4\textwidth}
  \centering
  \includegraphics[width=0.38\textwidth]{images/16_qam.png}
  \caption{Beispiel einer 16-QAM\cite{dig_nue}}
  \label{fig:qam_konstellation}
\end{wrapfigure}
Gerade bei neueren Mobilfunkstandards wie LTE und 5G ist die Quadrature Amplituden Moduation (QAM) weit verbreitet. Als Beispiel verwendet LTE eine 256-QAM Modulation, was bedeutet, dass 256 Symbole mit jeweils 8 Bit codiert werden. 
Dies grafisch darzustellen ist aufgrund der hohen Symbolanzahl schwierig, weshalb in Abbildung~\ref{fig:qam_konstellation} als Beispiel ein 16-QAM Konstellationsdiagramm gezeigt wird. \\
Mathematisch wird die QAM folgendermaßen beschrieben:
\begin{equation}
  s'_m(t) = I_m \cdot g(t) \ \text{mit} \ I_m = a_m + j \cdot b_m
  \label{eq:qam_modulation}
\end{equation}
\noindent Durch Gleichung~\ref{eq:qam_modulation} wird beschrieben, dass jedes Symbol $s'_m(t)$ durch eine Linearkombination aus einer komplexen Amplitude $I_m$ und einer Pulsformung $g(t)$ bestimmt wird. Die Pulsformung bestimmt hierbei über die Bandbreiteneffizienz. In einer wie hier gezeigten rechteckigen Konstellation haben alle Symbole den gleichen Abstand d zueinander. Dadurch, dass der minimale Abstand zwischen den direkt benachbarten Symbolen maximiert wird, 
ist die QAM effizient in der Bandbreitennutzung und gleichzeitig auch bei der Fehleranfälligkeit robust~\cite{wiley}. 